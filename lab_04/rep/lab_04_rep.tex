\documentclass[12pt]{report}

\usepackage{amsmath}
\usepackage{pgfplots}
\usepgfplotslibrary{units}
\usepackage[russian]{babel}
\usepackage{filecontents}
\usepackage{titlesec, blindtext, color}
\usepackage{listings}

\usepackage{titlesec, blindtext, color} 
\definecolor{gray75}{gray}{0.75} 
\newcommand{\hsp}{\hspace{20pt}} 

\usepackage{geometry}
\geometry{top=0.5cm}

\begin{document}
	
	\begin{titlepage}
		\centering
		{\scshape\LARGE МГТУ им. Баумана \par}
		\vspace{3cm}
		{\scshape\Large Лабораторная работа №4\par}
		\vspace{0.5cm}	
		{\scshape\Large По курсу: "Анализ алгоритмов"\par}
		\vspace{1.5cm}
		{\huge\bfseries Исследование работы алгоритма Винограда для умножения матриц реализованного при помощи параллельных вычислений\par}
		\vspace{2cm}
		\Large Работу выполнил: Подвашецкий Дмитрий, ИУ7-54Б\par
		\vspace{0.5cm}
		\LargeПреподаватели:  Волкова Л.Л., Строганов Ю.В.\par
		
		\vfill
		\large \textit {Москва, 2019} \par
	\end{titlepage}
	
	\tableofcontents
	
	\newpage
	\chapter*{Введение}
	\addcontentsline{toc}{chapter}{Введение}
	
	\textbf{Матрица} - математический объект, записываемый в виде прямоугольной таблицы элементов кольца или поля которая представляет собой совокупность строк и столбцов, на пересечении которых находятся её элементы.
	
	Матрицы широко применяются в математике для компактной записи систем линейных алгебраических или дифференциальных уравнений. В этом случае, количество строк матрицы соответствует числу уравнений, а количество столбцов — количеству неизвестных. В результате решение систем линейных уравнений сводится к операциям над матрицами.
	
	Матрицы допускают следующие алгебраические операции:
	\begin{enumerate}
		\item сложение матриц, имеющих один и тот же размер;
		\item умножение матриц подходящего размера;
		\item умножение матрицы на элемент основного кольца или поля;
	\end{enumerate}
	
	\textbf{Умножение матриц} - одна из основных операций над матрицами.
	
	Целью данной лабораторной работы является реализация и изучение алгоритма умножения матриц методом Винограда и стандартного.
	
	\textbf{Задачами} данной лабораторной являются:
	\begin{enumerate}
		\item оптимизировать алгоритм Винограда при помощи распараллеливания вычислений;
		\item исследовать поведение функции при обработке различным числом потоков. 
	\end{enumerate}
	
	\chapter{Аналитическая часть}
	В алгоритме умножения матриц методом Винограда каждый элемент производной матрицы считается как скалярное произведение.
	Рассмотрим два вектора:
	\begin{center}
		{$
			V = 
			\begin{pmatrix}
			v1 & v2 & v3 & v4
			\end{pmatrix}
			$}
		
		{$
			W = 
			\begin{pmatrix}
			w1 & w2 & w3 & w4
			\end{pmatrix}
			$}
	\end{center}
	
	Их скалярное произведение равно:
	\begin{equation}
	V\cdot W = v1w1 + v2w2 + v3w3 + v4w4
	\end{equation}
	
	Выражение (1.1) можно переписать:
	\begin{equation}
	V\cdot W = (v1 + w2)(v2 + w1) + (v3 + v4)(v4 + w3) - v1v2 - v3v4 - w1w2 - w3w4
	\end{equation}
	
	Можно заметить, что правую часть данного выражения можно высчитать заранее для каждого вектора.
	Это означает, что, при предварительной обработке векторов мы можем, в дальнейшем, сэкономить 2 операции умножения, за счет 2х лишних операций сложения.
	
	Если при умножении двух матриц произвести обработку строк первой и столбцов второй, то можно добиться большей эффективности по времени.
	
	\section*{Вывод}
	\addcontentsline{toc}{section}{Вывод}
	
	В данном разделе был описан алгоритм Винограда.
	
	
\end{document}