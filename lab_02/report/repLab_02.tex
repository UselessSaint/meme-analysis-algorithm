\documentclass[12pt]{report}

\usepackage{amsmath}
\usepackage{pgfplots}
\usepgfplotslibrary{units}
\usepackage[russian]{babel}
\usepackage{filecontents}
\usepackage{titlesec, blindtext, color}
\usepackage{listings}

\usepackage{titlesec, blindtext, color} 
\definecolor{gray75}{gray}{0.75} 
\newcommand{\hsp}{\hspace{20pt}} 

\lstset{ 
language=haskell,                 
basicstyle=\small\sffamily, 
numbers=left,              
numberstyle=\tiny,        
stepnumber=1,              
numbersep=5pt,             
showspaces=false,          
showstringspaces=false,   
showtabs=false,             
frame=single,            
tabsize=2,                
captionpos=t,              
breaklines=true,           
breakatwhitespace=false, 
escapeinside={\#*}{*)}   
}

\titleformat{\chapter}[hang]{\Huge\bfseries}{\thechapter\hsp\textcolor{gray75}{|}\hsp}{0pt}{\Huge\bfseries}

\begin{filecontents}{uMult.dat}
1000 0.00020209800000000003
1100 0.00021410339999999986
1200 0.0002370904000000001
1300 0.00029517500000000006
1400 0.0007464258000000018
1500 0.000584367800000001
1600 0.0005163561999999999
1700 0.00041921079999999994
1800 0.0003351362
1900 0.0003292243999999998
2000 0.00014309139999999984
\end{filecontents}

\begin{filecontents}{wMult.dat}
1000 0.0003251438000000001
1100 0.0003461451999999997
1200 0.00036822080000000005
1300 0.0004152217999999998
1400 0.0003882159999999997
1500 0.00041226099999999987
1600 0.0003832005999999999
1700 0.0004112356000000002
1800 0.0003932149999999998
1900 0.0004533189999999999
2000 0.0004923465999999997
\end{filecontents}

\begin{filecontents}{wMultU1.dat}
1000 5.999e-05
1100 4.0023999999999995e-05
1200 0.000120024
1300 0.000220244
1400 0.000119938
1500 0.00020029199999999997
1600 0.00016027399999999996
1700 0.00022008200000000003
1800 0.00022051799999999995
1900 0.000139946
2000 0.00026034399999999997
\end{filecontents}

\begin{filecontents}{wMultU2.dat}
1000 0.000140154
1100 6.0006e-05
1200 0.000120026
1300 0.00014006
1400 8.0012e-05
1500 8.001000000000001e-05
1600 0.000160208
1700 0.000160202
1800 0.00012006400000000001
1900 0.00014009
2000 0.00010008200000000002
\end{filecontents}

\begin{filecontents}{wMultU3.dat}
1000 8.003399999999999e-05
1100 6.0022e-05
1200 6.0134000000000004e-05
1300 0.00014006199999999996
1400 0.00014003599999999999
1500 0.00018030200000000002
1600 0.000140038
1700 0.00013985200000000002
1800 0.00012010399999999999
1900 0.00016003
2000 0.000400234
\end{filecontents}

\begin{document}

\begin{titlepage}
	\centering
	{\scshape\LARGE МГТУ им. Баумана \par}
	\vspace{3cm}
	{\scshape\Large Лабораторная работа №2\par}
	\vspace{0.5cm}	
	{\scshape\Large По курсу: "Анализ алгоритмов"\par}
	\vspace{1.5cm}
	{\huge\bfseries Умножение матриц\par}
	\vspace{2cm}
	\Large Работу выполнила: Подвашецкий Дмитрий, ИУ7-54Б\par
	\vspace{0.5cm}
	\LargeПреподаватели:  Волкова Л.Л., Строганов Ю.В.\par

	\vfill
	\large \textit {Москва, 2019} \par
\end{titlepage}

\tableofcontents

\newpage
\chapter*{Введение}
\addcontentsline{toc}{chapter}{Введение}

\textbf{Матрица} - математический объект, записываемый в виде прямоугольной таблицы элементов кольца или поля которая представляет собой совокупность строк и столбцов, на пересечении которых находятся её элементы.

Матрицы широко применяются в математике для компактной записи систем линейных алгебраических или дифференциальных уравнений. В этом случае, количество строк матрицы соответствует числу уравнений, а количество столбцов — количеству неизвестных. В результате решение систем линейных уравнений сводится к операциям над матрицами.

Матрицы допускают следующие алгебраические операции:
\begin{enumerate}
	\item сложение матриц, имеющих один и тот же размер;
	\item умножение матриц подходящего размера;
	\item умножение матрицы на элемент основного кольца или поля;
\end{enumerate}

\textbf{Умножение матриц} - одна из основных операций над матрицами.

Целью данной лабораторной работы является реализация и изучение алгоритма умножения матриц методом Винограда и стандартного.

Задачами данной лабораторной являются:
\begin{enumerate}
	\item изучение стандарного метода и метода Винограда для умножения матриц;
	\item реализация данных двух методов, а так же оптимизация последнего;
	\item теоретический анализ трудоемкости рассматриваемых алгоритмов;
	\item экспериментальное подтверждение различий во временнóй эффективности рассматриваемых алгоритмов;
	\item описание и обоснование полученных результатов в отчете о выполненной лабораторной
работе, выполненного как расчётно-пояснительная записка к работе.
\end{enumerate}

\chapter{Аналитическая часть}
Операция умножения матриц повсеместно приминяется в математике, физике, программировании и т.д.
Для того, чтобы произведение матрицы A, размерами n на m, на матрицу B, размерами u на v, было возможно, необходимо, чтобы n = u.

В данной лабораторной работе я рассмотрю два алгоритма умножения матриц.

Первый алгоритм - стандартный. 

Пусть есть две матрицы: 
\begin{center}
{$
A_{nm} = 
\begin{pmatrix}
  a_{00} &  ... & a_{0m}\\
   ... & ... & ...\\
  a_{n0} &  ... &  a_{nm}
\end{pmatrix}
$};
{$
B_{mk} = 
\begin{pmatrix}
  b_{00} &  ... & b_{0k}\\
   ... & ... & ...\\
  b_{m0} &  ... &  b_{mk}
\end{pmatrix}
$}
\end{center}

Тогда, пусть:
\begin{center}
{$
C_{nk} = A_{nm}B_{mk}
$} 
\end{center}
Где ij эллемент матрицы {$C_{nk}$} вычисляется как скалярное произведение i строки матрицы  {$A_{nm}$} на j столбец матрицы {$B_{mk}$}.

\begin{center}
{$C_{ij} = a_{0i} * b_{j0} + ... + a_{mi} * b_{jm},  i = [1..n], j = [1..m]$}
\end{center}

Второй алгоритм - Винограда.

В данном алгоритме, также как и в предыдущем, каждый элемент производной матрицы считается как скалярное произведение.
Рассмотрим два вектора:
\begin{center}
{$
V = 
\begin{pmatrix}
  v1 & v2 & v3 & v4
\end{pmatrix}
$}

{$
W = 
\begin{pmatrix}
  w1 & w2 & w3 & w4
\end{pmatrix}
$}
\end{center}

Их скалярное произведение равно:
\begin{center}
{$
V*W = v1w1 + v2w2 + v3w3 + v4w4
$}
\end{center}

Это выражение можно переписать:
\begin{center}
{$
V*W = (v1 + w2)(v2 + w1) + (v3 + v4)(v4 + w3) - v1v2 - v3v4 - w1w2 - w3w4
$}
\end{center}

Можно заметить, что правую часть данного выражения можно высчитать заранее для каждого вектора.
Это означает, что, при предварительной обработке векторов мы можем, в дальнейшем, сэкономить 2 операции умножения, за счет 2х лишних операций сложения.

Если при умножении двух матриц произвести обработку строк первой и столбцов второй, то можно добиться большей эффективности по времени.


















\chapter*{Список литературы}
\begin{enumerate}
	\item Умножение матриц. [Электронный ресурс] Режим доступа: http://www.algolib.narod.ru/Math/Matrix.html Последння дата обращения: 10.10.2019
\end{enumerate}


















\end{document}